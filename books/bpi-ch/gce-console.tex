\epi{``In Go, the code does exactly what it says on the page.''}{\textit{Go
Nuts mailing list}\\\textsc{ANDREW GERRAND}}
\noindent{}There are a few things that make Go different from other
languages.

\section{Getting Started with gcutil}

\subsection{Install gcutil}
\begin{enumerate}
\item check your machine internet connection
  \begin{lstlisting}[language=Bash]
tjyang@tj1210:~/gcep$ ping -c 2 google-compute-engine-tools.googlecode.com
PING googlecode.l.googleusercontent.com (74.125.142.82) 56(84) bytes of data.
64 bytes from ie-in-f82.1e100.net (74.125.142.82): icmp_req=1 ttl=128 time=25.9 ms
64 bytes from ie-in-f82.1e100.net (74.125.142.82): icmp_req=2 ttl=128 time=34.3 ms

--- googlecode.l.googleusercontent.com ping statistics ---
2 packets transmitted, 2 received, 0% packet loss, time 1002ms
rtt min/avg/max/mdev = 25.927/30.128/34.330/4.205 ms
tjyang@tj1210:~/gcep$ 
  \end{lstlisting}

\item if your machine is behind a http proxy firewall.
  \begin{lstlisting}[language=Bash]
tjyang@tj1210:~/gcep$ grep http /etc/wgetrc 
# You can set the default proxies for Wget to use for http, https, and ftp.
#https\_proxy = http://proxy.yoyodyne.com:18023/
#http\_proxy = http://proxy.yoyodyne.com:18023/
#ftp\_proxy = http://proxy.yoyodyne.com:18023/
tjyang@tj1210:~/gcep$ 
  \end{lstlisting}
\item Get gcutil-1.4.1.tar.gz
  \begin{lstlisting}[language=Bash]
   wget https://google-compute-engine-tools.googlecode.com/files/gcutil-1.4.1.tar.gz
  \end{lstlisting}
\item untar the python script and move it to /usr/local/bin
  \begin{lstlisting}[language=Bash]
tjyang@tj1210:~$ tar xzf gcutil-1.4.1.tar.gz 
tjyang@tj1210:~$ sudo mv gcutil-1.4.1 /usr/local/bin
[sudo] password for tjyang: 
tjyang@tj1210:~$ 
  \end{lstlisting}
\item make sure .bashrc in ~/.bash\_profile will be loaded after login.
  \begin{lstlisting}[language=Bash]
tjyang@tj1210:~$ cat .bash_profile 
if [ -f ~/.bashrc]; then source ~/.bashrc; fi
tjyang@tj1210:~$ 
  \end{lstlisting}
\item in .bashrc file, make sure gcutil and gsutil path are appended in PATH variable.
  \begin{lstlisting}[language=Bash]
TJYANG-MBA:~ tjyang$ 
echo ``export PATH=${PATH}:/usr/local/bin/gcutil-1.4.1:/usr/local/bin/gsutil'' >> ~/.bashrc
TJYANG-MBA:~ tjyang$ tail -1 ~/.bashrc
export PATH=${PATH}:/usr/local/bin/gcutil-1.4.1:/usr/local/bin/gsutil
TJYANG-MBA:~ tjyang$ 
  \end{lstlisting}

\item gcutil -version
  \begin{lstlisting}[language=Bash]
tjyang@tj1210:~/gcep$ which gcutil
/usr/local/bin/gcutil-1.4.1/gcutil
tjyang@tj1210:~/gcep$ gcutil version
1.4.1
tjyang@tj1210:~/gcep$ 
  \end{lstlisting}
\end{enumerate}

\subsection{GCE account authentication}
\begin{enumerate}
\item abc
  \begin{lstlisting}[language=Bash]
tjyang@tj1210:~/gcep$ gcutil auth --project=motorola.com:motogce 
tjyang@tj1210:~$ gcutil auth --project=motorola.com:motogce 
Go to the following link in your browser:

    https://accounts.google.com/o/oauth2/auth?scope=https%3A%2F%2Fwww.googleapis.com%2Fauth%2Fcompute+https%3A%2F%2Fwww.googleapis.com%2Fauth%2Fcompute.readonly+https%3A%2F%2Fwww.googleapis.com%2Fauth%2Fdevstorage.full_control+https%3A%2F%2Fwww.googleapis.com%2Fauth%2Fdevstorage.read_only+https%3A%2F%2Fwww.googleapis.com%2Fauth%2Fdevstorage.read_write+https%3A%2F%2Fwww.googleapis.com%2Fauth%2Fdevstorage.write_only+https%3A%2F%2Fwww.googleapis.com%2Fauth%2Fuserinfo.email&redirect_uri=urn%3Aietf%3Awg%3Aoauth%3A2.0%3Aoob&response_type=code&client_id=1025389682001.apps.googleusercontent.com&access_type=offline

Enter verification code: 4/039LJzdXr-ADrGnRgHzj7jwwt5wM.gnPgUIhM714euJJVnL49Cc_m9yJgdgI
Authentication successful.
INFO: Authorization succeeded for user tjyang@motorola.com
tjyang@tj1210:~$ ls -la .gcu*
-rw------- 1 tjyang tjyang 1547 Nov 23 19:26 .gcutil_auth
-rw-rw-r-- 1 tjyang tjyang   94 Nov 23 18:35 .gcutil.version
tjyang@tj1210:~$ 
tjyang@tj1210:~$ cat .gcutil.version 
{"current_version": "1.4.1", "last_checked_version": "1.4.1", "last_check": 1353717309.056243}tjyang@tj1210:~$ 
tjyang@tj1210:~$ cat .gcutil_auth 
{
  "data": [
    {
      "credential": {
        "_class": "OAuth2Credentials", 
        "_module": "oauth2client.client", 
        "access_token": "ya29.AHES6ZSdWfNOFZBKOjnZ4kjpG7F10uKWtof2TmvyOx9KmNQ", 
        "client_id": "1025389682001.apps.googleusercontent.com", 
        "client_secret": "xslsVXhA7C8aOfSfb6edB6p6", 
        "id_token": {
          "aud": "1025389682001.apps.googleusercontent.com", 
          "cid": "1025389682001.apps.googleusercontent.com", 
          "email": "tjyang@motorola.com", 
          "exp": 1353724642, 
          "hd": "motorola.com", 
          "iat": 1353720742, 
          "iss": "accounts.google.com", 
          "token_hash": "AqHvhwLHl76fcQyxSxUPKA", 
          "verified_email": "true"
        }, 
        "invalid": false, 
        "refresh_token": "1/KHb5N7KXOHZLlE0QNm9tuR2KGPr1eU0adVF_82aa1h4", 
        "token_expiry": "2012-11-24T02:37:22Z", 
        "token_uri": "https://accounts.google.com/o/oauth2/token", 
        "user_agent": "google-compute-cmdline/1.0"
      }, 
      "key": {
        "clientId": "1025389682001.apps.googleusercontent.com", 
        "scope": "https://www.googleapis.com/auth/compute https://www.googleapis.com/auth/compute.readonly https://www.googleapis.com/auth/devstorage.full_control https://www.googleapis.com/auth/devstorage.read_only https://www.googleapis.com/auth/devstorage.read_write https://www.googleapis.com/auth/devstorage.write_only https://www.googleapis.com/auth/userinfo.email", 
        "userAgent": "google-compute-cmdline/1.0"
      }
    }
  ], 
  "file_version": 1
}tjyang@tj1210:~$ 

  \end{lstlisting}
\end{enumerate}

\subsection{Simple check}
\begin{enumerate}
\item abe
  \begin{lstlisting}[language=Bash]

tjyang@tj1210:~$ gcutil --project=motorola.com:motogce auth --just_check_auth --cache_flag_values=True
INFO: Authorization succeeded for user tjyang@motorola.com
tjyang@tj1210:~$
tjyang@tj1210:~$ cat .gcutil.flags 
--project=motorola.com:motogce
--just_check_auth
tjyang@tj1210:~$ 
  \end{lstlisting}
\end{enumerate}

\section{gcutil list of commands}
 \lstset{basicstyle=\tiny\color{black}}
  \begin{lstlisting}[language=Bash]
Command line tool for interacting with Google Compute Engine.

Please refer to http://developers.google.com/compute/docs/gcutil/tips for more information about gcutil
usage.


USAGE: gcutil [--global_flags] <command> [--command_flags] [args]


Any of the following commands:
  addaccessconfig, adddisk, addfirewall, addimage, addinstance, addnetwork, auth, deleteaccessconfig,
  deletedisk, deletefirewall, deleteimage, deleteinstance, deletenetwork, deleteoperation, getdisk,
  getfirewall, getimage, getinstance, getkernel, getmachinetype, getnetwork, getoperation, getproject,
  getserialportoutput, getzone, help, listdisks, listfirewalls, listimages, listinstances, listkernels,
  listmachinetypes, listnetworks, listoperations, listzones, pull, push, setcommoninstancemetadata,
  ssh, version


addaccessconfig            Adds an access config to an instance's network interface.

                           Usage: gcutil [--global_flags] addaccessconfig [--command_flags]
                           <instance-name>

adddisk                    Create new machine disks.

                           More than one disk name can be specified. Multiple disks will be created in
                           parallel.

                           Usage: gcutil [--global_flags] adddisk [--command_flags] <disk-name-1> ...
                           <disk-name-n>

addfirewall                Create a new firewall rule to allow incoming traffic for instances on a
                           given network.

                           Usage: gcutil [--global_flags] addfirewall [--command_flags] <firewall-name>

addimage                   Create a new machine image.

                           The root_source_tarball parameter must point to a tar file of the contents
                           of the desired root directory stored in Google Storage.

                           Usage: gcutil [--global_flags] addimage [--command_flags] <image-name>
                           <root-source-tarball>

addinstance                Create new VM instances.

                           More than one instance name can be specified. Multiple instances will be
                           created in parallel.

                           Usage: gcutil [--global_flags] addinstance [--command_flags]
                           <instance-name-1> ... <instance-name-n>

addnetwork                 Create a new network instance.

                           Usage: gcutil [--global_flags] addnetwork [--command_flags] <network-name>

auth                       Class for forcing client authorization.

                           Usage: gcutil [--global_flags] auth [--command_flags]

deleteaccessconfig         Deletes an access config from an instance's network interface.

                           Usage: gcutil [--global_flags] deleteaccessconfig [--command_flags]
                           <instance-name>

deletedisk                 Delete one or more machine disks.

                           If multiple disk names are specified, the disks will be deleted in parallel.

                           Usage: gcutil [--global_flags] deletedisk [--command_flags] <disk-name-1>
                           ... <disk-name-n>

deletefirewall             Delete one or more firewall rules.

                           If multiple firewall names are specified, the firewalls will be deleted in
                           parallel.

                           Usage: gcutil [--global_flags] deletefirewall [--command_flags]
                           <firewall-name-1> ... <firewall-name-n>

deleteimage                Delete one or more machine images.

                           If multiple image names are specified, the images will be deleted in
                           parallel.

                           Usage: gcutil [--global_flags] deleteimage [--command_flags] <image-name-1>
                           ... <image-name-n>

deleteinstance             Delete one or more VM instances.

                           If multiple instance names are specified, the instances will be deleted in
                           parallel.

                           Usage: gcutil [--global_flags] deleteinstance [--command_flags]
                           <instance-name-1> ... <instance-name-n>

deletenetwork              Delete one or more machine networks.

                           If multiple network names are specified, the networks will be deleted in
                           parallel.

                           Usage: gcutil [--global_flags] deletenetwork [--command_flags]
                           <network-name-1> ... <network-name-n>

deleteoperation            Delete one or more operations.

                           Usage: gcutil [--global_flags] deleteoperation [--command_flags]
                           <operation-name-1> ... <operation-name-n>

getdisk                    Get a machine disk.

                           Usage: gcutil [--global_flags] getdisk [--command_flags] <disk-name>

getfirewall                Get a firewall.

                           Usage: gcutil [--global_flags] getfirewall [--command_flags] <firewall-name>

getimage                   Get a machine image.

                           Usage: gcutil [--global_flags] getimage [--command_flags] <image-name>

getinstance                Get a machine instance.

                           Usage: gcutil [--global_flags] getinstance [--command_flags] <instance-name>

getkernel                  Get a kernel.

                           Usage: gcutil [--global_flags] getkernel [--command_flags]

getmachinetype             Get a machine type.

                           Usage: gcutil [--global_flags] getmachinetype [--command_flags]

getnetwork                 Get a network instance.

                           Usage: gcutil [--global_flags] getnetwork [--command_flags] <network-name>

getoperation               Retrieve an operation resource.

                           Usage: gcutil [--global_flags] getoperation [--command_flags]
                           <operation-name>

getproject                 Get the resource for a Google Compute Engine project.

                           Usage: gcutil [--global_flags] getproject [--command_flags] <project-name>

getserialportoutput        Get the output of an instance's serial port.

                           Usage: gcutil [--global_flags] getserialportoutput [--command_flags]
                           <instance-name>

getzone                    Get a zone.

                           Usage: gcutil [--global_flags] getzone [--command_flags] <zone-name>

help                       Help for all or selected command:
                               gcutil help [<command>]

                           To retrieve help with global flags:
                               gcutil --help

                           To retrieve help with flags only from the main module:
                               gcutil --helpshort [<command>]

listdisks                  List the machine disks for a project.

                           Usage: gcutil [--global_flags] listdisks [--command_flags]

listfirewalls              List the firewall rules for a project.

                           Usage: gcutil [--global_flags] listfirewalls [--command_flags]

listimages                 List the machine images for a project.

                           Usage: gcutil [--global_flags] listimages [--command_flags]

listinstances              List the machine instances for a project.

                           Usage: gcutil [--global_flags] listinstances [--command_flags]

listkernels                List the machine kernels for a project.

                           Usage: gcutil [--global_flags] listkernels [--command_flags]

listmachinetypes           List the machine types.

                           Usage: gcutil [--global_flags] listmachinetypes [--command_flags]

listnetworks               List the machine networks for a project.

                           Usage: gcutil [--global_flags] listnetworks [--command_flags]

listoperations             List the operations for a project.

                           Usage: gcutil [--global_flags] listoperations [--command_flags]

listzones                  List available zones.

                           Usage: gcutil [--global_flags] listzones [--command_flags]

pull                       Pull one or more files from an instance.

                           Usage: gcutil [--global_flags] pull [--command_flags] <instance-name>
                           <file-1> ... <file-n> <destination>

push                       Push one or more files to an instance.

                           Usage: gcutil [--global_flags] push [--command_flags] <instance-name>
                           <file-1> ... <file-n> <destination>

setcommoninstancemetadata  Set the commonInstanceMetadata field for a Google Compute Engine project.

                           This is a blanket overwrite of all of the project wide metadata.

                           Usage: gcutil [--global_flags] setcommoninstancemetadata [--command_flags]

ssh                        Ssh to an instance.

                           Usage: gcutil [--global_flags] ssh [--command_flags] <instance-name>
                           <ssh-args>

version                    Get the current version of this command.

                           Usage: gcutil [--global_flags] version [--command_flags]


Run 'gcutil --help' to get help for global flags.
Run 'gcutil help <command>' to get help for <command>.
  \end{lstlisting}

\section{addaccessconfig}
\begin{lstlisting}[language=Bash]
\end{lstlisting}

\section{adddisk}
\begin{lstlisting}[language=Bash]
\end{lstlisting}

\section{addfirewall}
\begin{lstlisting}[language=Bash]
\end{lstlisting}

\section{addimage}
\begin{lstlisting}[language=Bash]
\end{lstlisting}

\section{addinstance}
\begin{lstlisting}[language=Bash]
\end{lstlisting}

\section{addnetwork}
\begin{lstlisting}[language=Bash]
\end{lstlisting}

\section{auth}
\begin{lstlisting}[language=Bash]
\end{lstlisting}

\section{deleteaccessconfig}
\begin{lstlisting}[language=Bash]
\end{lstlisting}

\section{deletedisk}
\begin{lstlisting}[language=Bash]
\end{lstlisting}

\section{deletefirewall}
\begin{lstlisting}[language=Bash]
\end{lstlisting}

\section{deleteimage}
\begin{lstlisting}[language=Bash]
\end{lstlisting}

\section{deleteinstance}
\begin{lstlisting}[language=Bash]
\end{lstlisting}

\section{deletenetwork}
\begin{lstlisting}[language=Bash]
\end{lstlisting}

\section{deleteoperation}
\begin{lstlisting}[language=Bash]
\end{lstlisting}

\section{getdisk}
\begin{lstlisting}[language=Bash]
\end{lstlisting}

\section{getfirewall}
\begin{lstlisting}[language=Bash]
\end{lstlisting}

\section{getimage}
\begin{lstlisting}[language=Bash]
\end{lstlisting}

\section{getinstance}
\begin{lstlisting}[language=Bash]
\end{lstlisting}

\section{getkernel}
\begin{lstlisting}[language=Bash]
\end{lstlisting}

\section{getmachinetype}
\begin{lstlisting}[language=Bash]
\end{lstlisting}

\section{getnetwork}
\begin{lstlisting}[language=Bash]
\end{lstlisting}

\section{getoperation}
\begin{lstlisting}[language=Bash]
\end{lstlisting}

\section{getproject}
\begin{lstlisting}[language=Bash]
\end{lstlisting}

\section{getserialportoutput}
\begin{lstlisting}[language=Bash]
\end{lstlisting}

\section{getzone}
\begin{lstlisting}[language=Bash]
\end{lstlisting}

\section{help}
\begin{lstlisting}[language=Bash]
\end{lstlisting}

\section{listdisks}
\begin{lstlisting}[language=Bash]
\end{lstlisting}

\section{listfirewalls}
\begin{lstlisting}[language=Bash]
\end{lstlisting}

\section{listimages}
\begin{lstlisting}[language=Bash]
\end{lstlisting}

\section{listinstances}
\begin{lstlisting}[language=Bash]
\end{lstlisting}

\section{listkernels}
\begin{lstlisting}[language=Bash]
\end{lstlisting}

\section{listmachinetypes}
\begin{lstlisting}[language=Bash]
\end{lstlisting}

\section{listnetworks}
\begin{lstlisting}[language=Bash]
\end{lstlisting}

\section{listoperations}
\begin{lstlisting}[language=Bash]
\end{lstlisting}

\section{listzones}
\begin{lstlisting}[language=Bash]
\end{lstlisting}

\section{pull}
\begin{lstlisting}[language=Bash]
\end{lstlisting}

\section{push}
\begin{lstlisting}[language=Bash]
\end{lstlisting}

\section{setcommoninstancemetadata}
\begin{lstlisting}[language=Bash]
\end{lstlisting}

\section{ssh}
\begin{lstlisting}[language=Bash]
\end{lstlisting}

\section{version}
\begin{lstlisting}[language=Bash]
\end{lstlisting}


